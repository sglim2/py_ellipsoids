\documentclass[11pt]{article}
\usepackage{verbatim}
\title{\textbf{Py\_Ellipsoids}}
\author{Ian Merrick\\
		School of Earth and Ocean Sciences\\
		Cardiff University\\
		Wales, UK}
\date{}
\begin{document}

\maketitle

\section{Overview}
The aim of this code is to represent any second order tensor as an ellipsoid for use in virtual globes (i.e. google-earth).

The code is written in python and at its core uses the PyCollada\cite{pycollada} and SimpleKML\cite{simplekml} modules. The code takes a list of defined ellipsoids and creates a zipped KML\cite{kml} file with the ellipsoids embedded as COLLADA\cite{collada} objects.

The code is tested with Python versions $2.7+$ and $3.4+$.

\section{Installation}
Check out the repository from bitbucket ({\em https://bitbucket.org/sglim2/py\_ellipsoids}). Assuming you already have python and python-pip installed and in your path, you can install all required modules with the command:
\begin{verbatim}
 # pip install -r requirements.txt
\end{verbatim}

\section{Usage}
To run the application, from the command-line type:
\begin{verbatim}
 # python py_ellipsoids.py [-r RESOLUTION] [--keep] input output
\end{verbatim}
where {\em input} is the ellipsoid definition file, {\em output} is the resulting KMZ file, and {\em RESOLUTION} is the desired resolution of the ellipsoid grid. {\em input} and {\em output} are required arguments, {\em RESOLUTION} is optional and defaults to $16$.

\section{The Ellipsoid Definition File}
An example ellipsoid definition file is given:
\verbatiminput{../src/ellipsoids_example3.csv}
It is a comma-separated-value (csv) file with headings. The columns of the csv file may be in any order. A list of the required headings are given (with descriptions):
\begin{table}[h]
\begin{tabular}{rl}
 description & (The name or short description of the ellipsoid/tensor)  \\
 A           & (One of the 3 ellipsoid semi-axes lengths)  \\
 B           & (One of the 3 ellipsoid semi-axes lengths)  \\
 C           & (One of the 3 ellipsoid semi-axes lengths)  \\
 lat         & (Global latitude position of the centre of the ellipsoid)  \\
 lon         & (Global longitude position of the centre of the ellipsoid) \\
 alt         & (Altitude of the ellipsoid relative to the ground at lat/lon)  \\
 alpha       & (Angle in degrees of the pitch of the ellipsoid)  \\
 beta        & (Angle in degrees of the trend of the ellipsoid)  \\
 gamma       & (Angle in degrees of the roll of the ellipsoid)  \\
 colour      & (One of any pre-defined colours)  \\
 transparency& (The level of transparency of the ellipsoid (range 0..1))
\end{tabular}
\end{table}

\subsection{Position and Orientation of the Ellipsoids}






\begin{thebibliography}{1}

  \bibitem{pycollada} PyCollada ({\em https://github.com/pycollada/pycollada}).

  \bibitem{simplekml} SimpleKML ({\em https://code.google.com/p/simplekml}).
  
  \bibitem{kml} KML ({\em http://www.opengeospatial.org/standards/kml}).
  
  \bibitem{collada} COLLADA ({\em https://collada.org/}).

\end{thebibliography}

\end{document}
